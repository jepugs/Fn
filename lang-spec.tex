% Created 2021-12-12 Sun 20:55
% Intended LaTeX compiler: pdflatex
\documentclass[11pt]{article}
\usepackage[utf8]{inputenc}
\usepackage[T1]{fontenc}
\usepackage{graphicx}
\usepackage{grffile}
\usepackage{longtable}
\usepackage{wrapfig}
\usepackage{rotating}
\usepackage[normalem]{ulem}
\usepackage{amsmath}
\usepackage{textcomp}
\usepackage{amssymb}
\usepackage{capt-of}
\usepackage{hyperref}
\author{Jack Pugmire}
\date{\today}
\title{Fn Language Specification}
\hypersetup{
 pdfauthor={Jack Pugmire},
 pdftitle={Fn Language Specification},
 pdfkeywords={},
 pdfsubject={},
 pdfcreator={Emacs 27.2 (Org mode 9.4.4)}, 
 pdflang={English}}
\begin{document}

\maketitle
\tableofcontents


\section{About this Document}
\label{sec:orge8c6f6a}

This document fully specifies the syntax and semantics of the Fn programming
language. Any part of it may be changed at any time. Since the interpreter is
still unfinished, its actual behavior may differ from what's described here.

Section \ref{sec:orgbc0d4fa} contains a description of the semantics of Fn. It's
meant to read like a user's manual, and should cover everything about the core
language when it's done. I'm trying to make this readable.

After this comes the technical part of the document. It's a big old list of all
the types of expressions in Fn with precise descriptions of their behaviors.

Following this is the lexical analysis section. This rounds out the
specification by giving a character-by-character description of Fn syntax. I'm
planning to rewrite this at some point. It's in rough shape after repeated
cut-and-pastes from different documents.

At the end is some other miscellaneous stuff I copied from old design documents
because I didn't want to delete it from the main repository yet.

There will be more sections (and subsections!) in the future.


\section{Language Description}
\label{sec:orgbc0d4fa}

\subsection{Overview}
\label{sec:org91300c7}

Fn is a dynamically typed, garbage-collected, general-purpose programming
language with Lisp-like syntax. A couple of its defining characteristics are:
\begin{itemize}
\item The core language has relatively few operators, with special care taken to
ensure they have consistent and concise syntax.
\item Fn is designed for a "mostly functional" programming style, where immutability
and referential transparency are favored, but not required.
\item The native macro system allows definition of new syntax (and even entire
domain-specific languages) using regular Fn code and data structures.
\end{itemize}

The syntax looks like this:

\begin{verbatim}
(def sqrt-precision 0.001)
(def approx-sqrt (x)
  (letfn iterate (guess)
    (if (< (abs (- x (* guess guess)) sqrt-precision))
        guess
        (iterate (/ (+ guess (/ x guess) 2)))))
  (if (>= x 0)
      (iterate (/ x 2))
      (error "Cannot approximate square root of a negative number.")))
\end{verbatim}


\subsection{Variables and Mutation}
\label{sec:orga508ab5}

\subsubsection{Local Variables}
\label{sec:orgc302283}

Local variables can be created using one of the special operators \texttt{let}, \texttt{letfn}, or
\texttt{with}.\footnote{In Fn, the \texttt{with} operator provides the functionality of what
most Lisp-like languages call \texttt{let}, while Fn's \texttt{let} is quite different, as it acts
on the surrounding environment.} They all bind variables in the same way, but with
different syntax for programmer convenience. Function parameters are also
treated as local variables within the function body.

Before proceeding, we note that the full story about local variables involves
variable capture semantics, which are covered in a later section. Variable
capture doesn't affect any of the concepts discussed in the rest of this section.

\texttt{let} is the most elementary way to create a local variable. It defines one or
more new variables in the current lexical environment.

\begin{verbatim}
;; let binds variables to the given values
(let x 'symbol)
;; multiple definitions can be made in a single let
(let a 16
     b (reverse "string")
     ;; value expressions can refer to variables from earlier in the same let
     c (+ a (length b))
\end{verbatim}

\texttt{with} is similar to \texttt{let}, but rather than creating definitions in the containing
environment, it creates a new lexical environment.

\begin{verbatim}
;; this creates two variables
(with (a 3
       b 4)
  ;; the body can contain multiple expressions
  (println "hello")
  (+ a b))
;; returns 7
;; the variables a, b do not exist outside of the with body
\end{verbatim}

\texttt{letfn} has a streamlined syntax for creating functions, but otherwise behaves
like \texttt{let}. See the documentation below for details.

All local variables can have their value changed with \texttt{set!}. The exclamation
point is because mutation is not to be taken lightly. The syntax for \texttt{set!} is
like this:

\begin{verbatim}
(set! var-name new-value)
;; for example
(let var 'hi)
(println var) ;; prints 'hi
(set! var 'lo)
(println var) ;; prints 'lo
\end{verbatim}

Note that attempting to \texttt{set!} a global variable will result in an error.


\subsubsection{Global Variables}
\label{sec:org25d2c11}

Global variables in Fn are created using \texttt{def} or \texttt{defn}. E.g.

\begin{verbatim}
(def my-global 'special-constant)
(def my-other-global (+ 21 69))
\end{verbatim}

\texttt{defn} behaves exactly like \texttt{def}, but has special syntax streamlined for defining
functions.

Global variables are immutable, i.e. they cannot be changed by using \texttt{set!}.
However, by assigning global variables to mutable datatypes or by exploiting
variable capture (discussed in a later section), mutable state can still be
associated to a global variable. This is intended behavior, however, it is not
recommended that you abuse it.


\subsection{Data Types}
\label{sec:orgd17d65e}

Fn provides the following builtin data types (type names in Fn are
\texttt{Capitalized-Like-This}):

\begin{description}
\item[{\texttt{Nil}}] The special constant \texttt{nil}, used to indicate no value.
\item[{\texttt{Bool}}] The special boolean constants \texttt{true} and \texttt{false}.
\item[{\texttt{Num}}] Floating-point numbers. (These are almost IEEE 64-bit floats, but we
truncate the significand by four bits to fit type information).
\item[{\texttt{Symbol}}] Internalized strings. These are essentially strings with a faster
equality test, at the expensive of slower access to the characters of the
string. They are used extensively by the macro system.
\item[{\texttt{String}}] (Immutable) sequences of bytes. Usually these are UTF-8 encoded
character streams.
\item[{\texttt{List}}] (Immutable) singley-linked lists.
\item[{\texttt{Table}}] Mutable key-value stores.
\end{description}

Of these, only lists and tables logically contain other values. (Substrings can
be extracted from strings, but this actually creates a new string object and
just copies in data from the other string). So, we call \texttt{List} and \texttt{Table} the two
\textbf{compound data types}, and call the rest of them \textbf{simple data types}.

\subsubsection{Simple Data Types}
\label{sec:org10b3e3a}
Here is what the syntax looks like for the simple data types:

\begin{verbatim}
;; numbers are pretty much what you'd expect
2
-6
3.14159
2.0e-6 ;; we have scientific notation
0xFf ;; hexadecimal, even!

;; strings are enclosed within matched double quotes
"string"
"Fn uses escape codes from C, e.g. \\ \"\n"
""

;; symbols are prefixed by a single quote.
'sym1
'sym2
;; symbols can contain whitespace and syntax characters, provided they are 
;; escaped with a backslash
'sym\ with\ \"escapes\"
;; be careful about the quote operator. If the quoted expression is a number,
;; it will result in a number instead of a symbol. You can get around this 
;; with escapes:
'0xb8  ;; this is a number
'\0xb8 ;; this is a symbol

;; booleans and nil are called by name
true
false
nil
\end{verbatim}

See also subsection \ref{sec:orgdf14b97} for more on symbols and the quote operator.


\subsubsection{Lists}
\label{sec:orga48c356}

Lists in Fn are what you'd expect for a functional programming language. They're
created using square brackets or by using the \texttt{List} function.

\begin{verbatim}
[] ; empty list
['a 'b] ; list of two symbols
[1 'a "str"] ; lists may contain objects of arbitrary type

;; List is identical to square bracket syntax
[1 2 3]
(List 1 2 3)
\end{verbatim}

Lists can be manipulated with builtin functions:

\begin{verbatim}
(def list1 [["str" 2] 'a 'b])
(def list2 [0 2 4 6 8 10])

;; head and tail access the head and tail of the list
(head list1) ;=> ["str" 2]
(head list2) ;=> 0
(tail list1) ;=> ['a 'b]

(tail [])    ;=> []
(head [])    ;=> error (empty list has no head)

;; nth allows random access:
(nth list1 2) ;=> 'b
(nth list2 1) ;=> 2

;; length gives the length of a list
(length []) ;=> 0
(length list1) ;=> 3
(length list2) ;=> 6

;; cons prepends elements
(cons 2 []) ;=> [2]
(cons nil list1) ;=> [nil ["str" 2] 'a 'b]

;; concat concatenates two or more lists
(concat [1 2 3] [4 5 6]) ;=> [1 2 3 4 5 6]
(concat [37] ['foo] ["bar"]) ;=> [37 'foo "bar"]
(concat list2 list1) ;=> [0 2 4 6 8 10 ["str" 2] 'a 'b]

;; reverse reverse the direction of a list
(reverse list2) => [10 8 6 4 2 0]
\end{verbatim}


\subsubsection{{\bfseries\sffamily TODO} Tables}
\label{sec:org4014214}

Tables are key-value stores. Any type of object may be used as a key or a value,
(note, however, that it takes longer to hash more complicated data structures
since we have to descend on their fields)\footnote{Two keys are equal if \texttt{(= k1 k2)} is true (using the builtin
equality function). For simple data types the meaning of equality is obvious.
Lists and tables are compared componentwise. That is, two lists are equal if and
only if all their respective entries are equal. Two tables are equal if their
key sets are equal (disregarding order), and for each key the corresponding
values in each table are equal.}.

Tables are built using braces \texttt{\{\}} or the equivalent \texttt{Table} function. This must be
passed an even number of arguments.

\begin{verbatim}
{} ;=> empty table
{'key1 4 'key 6} ;=> table with two kv-pairs
(Table 'key1 4 'key 6) ;=> table with two kv-pairs
\end{verbatim}

Table elements may be accessed using the builtin function \texttt{get}. When the key is a
constant symbol, dot syntax (or the equivalent \texttt{dot} special operator) can be
used instead. This is how this looks:

\begin{verbatim}
(def tab1 {'name "Mr. Table"
          'occupation "Holds data"
          'child {'name "Table Jr." 
                  'occupation "Holds less data"}})
(def tab2 {0 'zero 1 'one 2 'two 3 'three 4 'four})

;; these all return "Mr. Table"
(get tab1 'name)
tab1.name
(dot tab1 name) ; equivalent syntax to the dot expression
;; Note that the symbols in the dot expressions are unquoted. Arguments to dot 
;; must be unquoted symbols or a compilation error occurs.

;; get is more flexible than dot and allows arbitrary key and value expressions
(get tab2 (+ 1 2)) ;=> 'three
(get {'k 'v} 'k) ;=> 'v

;; dot makes it convenient to descend on tables with symbolic key names
tab1.child.name ;=> "Table Jr."
;; equivalent expression:
(dot tab1 child name)
\end{verbatim}

Since tables are mutable, the main way to populate them is to use the \texttt{set!}
operator (the same one as for local variables). In this case, the first argument
may be any legal \texttt{get} or \texttt{dot} expression on a table.

Lastly, tables size can be checked with \texttt{length}, a list of keys can be retrieved
with \texttt{table-keys}, and two or more tables can be combined with \texttt{concat} (if any of
the tables have keys in common, the last table in the argument list takes
priority).


\subsubsection{Quoting}
\label{sec:orgdf14b97}

"Quoting" refers to the process of converting Fn source code into native Fn
data. This allows us to easily process and manipulate Fn source code using the
same facilities as for normal data.

Quoting is the secret sauce that makes Fn's macro system work. It's the main
reason why Fn has the syntax it has.

The \texttt{quote} special operator has syntax:
\begin{verbatim}
(quote <expr>) ;; or, equivalently
'<expr>
\end{verbatim}
where \texttt{<expr>} can be any expression (in fact, it need not be a legal expression
by itself). These two notations are exactly the same. The interpreter expands
the second into the first before evaluation.

The value returned by quote is guaranteed to only consist of lists, symbols,
numbers, and strings. We refer to the latter three as \textbf{atoms}. Here are some
examples:
\begin{verbatim}
'(a b c) ;; returns ['a 'b 'c]
'"string" ;; returns "string"
'(+ a (/ x 2)) ;; returns ['+ 'a ['/ 'x 2]]

''quot ;; is equivalent to
(quote (quote quot)) ;; which returns ['quote 'quot]
\end{verbatim}

Note that \texttt{<expr>} only needs to be syntactically valid (i.e. not freak out the
parser). Illegal expressions can be quoted just fine:
\begin{verbatim}
'() ;; returns [] (the empty list)
'(2 (3 4)) ;; returns [2 [3 4]]
'(quote) ;; returns ['quote]
\end{verbatim}

This makes \texttt{quote} very handy for creating nested lists of atoms. (\texttt{quote} also has
a big sister named \texttt{quasiquote}, which is covered in the section on macros, and
allows for much more flexibility).

\texttt{quote} is also the primary way to create symbols. As noted in subsection \ref{sec:org10b3e3a}, this can lead to problems when we want a symbol whose name is a
syntactically valid number. Adding an escape character to the symbol name
designates to the parser that the token should be read as a symbol rather than a
number. In fact, we can even use this trick to give variables numbers for names:
\begin{verbatim}
;; probably don't do this
(def \2 3)
2  ;; returns 2
\2 ;; returns 3
\end{verbatim}

My recommendation: just don't use symbol names that are syntactically legal
numbers.


\subsection{{\bfseries\sffamily TODO} Control Flow and Functions}
\label{sec:org67814c5}

\subsubsection{Conditional Execution}
\label{sec:org0a9ff87}

The conditional control flow primitives are \texttt{if} and \texttt{cond}.


\subsubsection{Creating Functions}
\label{sec:org87d2bc1}

Functions are created using \texttt{fn}.

A short syntax is also provided for creating functions via the dollar sign,
which expands into a \texttt{dollar-fn} special form.

For example:
\begin{verbatim}
(fn (x) (* x x))
$(* $ $)
(dollar-fn (* $ $))
\end{verbatim}

All three of the above take in a single argument and square it. Note that
dollar-fn uses \texttt{\$} (or equivalently, \texttt{\$0}) for the name of the first parameter.
(Other positional parameters can be accessed with \texttt{\$1}, \texttt{\$2}, and so on). See
subsection \ref{sec:org975cdcb} for more details.

\texttt{fn} on the other hand has an explicit parameter list. The syntax for parameter
lists is this:
\begin{verbatim}
param-list      ::=  '(' <req-param>* <opt-param>* <var-params>? ')'
req-param       ::= <identifier>
opt-param       ::= (<identifier> <init-form>)
var-params      ::= <var-list-param> <var-table-param>?
                  | <var-table-param> <var-list-param>?
var-list-param  ::= '&' <identifier>
var-table-param ::= ':&' <identifier>
\end{verbatim}

In other words, parameter lists consist of zero or more required parameters,
zero or more optional parameters, and optionally end with variadic table and
list arguments.

Each of these parameters has an associated identifier (i.e. a symbol that is a
legal name). In the function's body, the respective arguments are bound to these
names. See subsection \ref{sec:org8352aea} for information about how argument lists
are processed during functino calls.


\subsubsection{Function Calls}
\label{sec:org8352aea}

Fn allows arguments to be named in function calls very similarly to Python.
Named arguments are passed using keywords, which are simply symbols whose names
begin with \texttt{:}. These symbols are not legal identifiers, so their appearance in
function calls is unambiguous. We also place the restriction that positional
arguments may not follow named ones. (Believe me, I tried to make it work
without that, and it's a mess at every level).

First we will deal with the case where there are no variadic parameters. See the
following example.
\begin{verbatim}
;; this function has 3 positional parameters, the last of which is optional
(defn arg-demo (x y (z 2))
  (* z (+ x y)))

;; here are a couple of ways we could call this function
(arg-demo 2 3)            ; x = 2, y = 3, z = 2, result = 10
(arg-demo 2 3 4)          ; x = 2, y = 3, z = 4, result = 20
(arg-demo :x 2 :y 3)      ; x = 2, y = 3, z = 2, result = 10
(arg-demo :z 2 :y 3 :x 2) ; x = 2, y = 3, z = 2, result = 10
(arg-demo :z 3 1 2)       ; error! positional argument following named argument
\end{verbatim}

To be precise, function parameters (still considering the case where there are
no variadic parameters) are bound using the following procedure:
\begin{itemize}
\item the unnamed arguments are bound to positional parameters in order
\item the named arguments are bound to their respective parameters, raising an error
if any duplicates or unrecognized names are found
\item unbound optional parameters are set to their default values. If any required
parameters remain unbound, an error is raised
\end{itemize}

Now, variadic arguments change some of the rules. We have two types of variadic
parameters in Fn: variadic tables, and variadic lists.

For tables, the semantics are very simple. Functions with a variadic table
parameter can accept any named argument, not just the names corresponding to
their functions (duplicated names are still not allowed). Moreover, it's now
possible for a named argument to have the same name as a positional argument.
\begin{verbatim}
;; demo function ignores first arg and returns table
(def var-tab-demo ((x nil) :& tab) ; variadic table arguments denoted with :&
  tab)

(var-tab-demo 0)         ; result = {}
(var-tab-demo :y 2 :x 1) ; result = {'y 1}
(var-tab-demo 1 :x 2)    ; result = {'x 2}
\end{verbatim}

As can be seen above, the variadic table is constructed by taking all
unrecognized named arguments and inserting them into the table as key-value
pairs. Moreover, if a named argument is recognized, but was already provided as
a positional argument, then that goes to the variadic table as well.

Variadic lists are analogous to variadic tables, but where those act on trailing
named arguments, variadic lists act on trailing positional arguments. As such,
it is impossible to use named arguments while at the same time passing a
non-empty variadic list argument, except in the case where there is also a
variadic table parameter to catch the trailing arguments.
\begin{verbatim}
;; demo function ignores first arg and returns list
(def var-lst-demo (x & list) ; variadic lists denoted with &
  list)

(var-lst-demo 0 1 2)  ;=> [1 2]
(var-lst-demo 0)      ;=> []
(var-lst-demo 0 :x 2) ;=> syntax error
(var-lst-demo :x 0)   ;=> []
(var-lst-demo :x 0 1) ;=> syntax error

;; demo function using both variadic parameters
(def var-mixed-demo (x & list :& table)
  [list table])

;; names not explicitly in the parameter list get sent to the variadic table
(var-mixed-demo :x 4 :y 2) ;=> [[] {'y 2}]
;; with a variadic table argument, duplicate names are allowed if one is a 
;; positional arg:
(var-mixed-demo 'a 'b :x 4 :y 2) ;=> [['b] {'x 4 'y 2}]
;; as always, keywords cannot precede positional arguments
(var-mixed-demo :x 4 :y 2 'a 'b) ;=> syntax error
\end{verbatim}


\subsubsection{Variable Capture}
\label{sec:org9945995}


\subsubsection{dollar-fn}
\label{sec:org975cdcb}

\subsection{{\bfseries\sffamily TODO} Namespaces and Import}
\label{sec:org4b0fe4d}

\subsubsection{Namespaces and Packages}
\label{sec:orgdeda7f2}

All code in Fn runs inside some namespace, which is used to hold currently
visible global variables and macros.

A \textbf{namespace} is a collection of macro and variable definitions. Namespaces are
identified by a \textbf{name}, which is a string not containing any slashes, and a
\textbf{package}, which is a string representing a logical collection of namespaces.
Finally, the symbol \texttt{<package>/<name>} is called the \textbf{identifier} or \textbf{ID} of the
namespace. This ID is required to be globally unique.

\textbf{Examples of Namespace IDs:}
\begin{verbatim}
fn/builtin              ; package is "fn", name is "builtin"
fn/internal/io          ; package is "fn/internal", name is "io"
my-project/util/linalg  ; package is "my-project/util", name is "linalg"
my-project/model        ; package is "my-project", name is "model"
\end{verbatim}

When evaluating code from a file, the namespace name will always be the stem of
the file. The package can be set via a package declaration, see \ref{sec:orgb5d06fc}.

The default REPL namespace is \texttt{fn/interactive}. Fn source code passed in as a
command line argument is also evaluated in this namespace.


\subsubsection{Import}
\label{sec:org85057e2}

The \texttt{import} special form allows definitions from an external namespace to be
copied into the current one. The syntax for import looks like this:

\begin{verbatim}
(import <namespace-id>)                  ; invocation 1
(import <namespace-id> :as <alias>)      ; invocation 2
(import <namespace-id> :no-prefix true)  ; invocation 3
\end{verbatim}

Say we have a namespace \texttt{foo/bar/baz} containing variables named \texttt{bob} and
\texttt{alice}:
\begin{verbatim}
;;; baz.fn
(package foo/bar)
(def alice "Alice")
(def bob "Bob")
\end{verbatim}

We have three ways to import this namespace, shown above. All three cause the
definitions from \texttt{foo/bar/baz} to be copied into the current namespace. However,
in each case the created bindings will have different names. The three cases are
illustrated below:

\begin{verbatim}
;;; main.fn

;; invocation 1
(import foo/bar/baz)
; variables look like this:
baz:alice
baz:bob

;; invocation 2
(import foo/bar/baz :as b)
; variables look like this:
b:alice
b:bob

;; invocation 3
(import foo/bar/baz :no-prefix true)
; variables are imported directly (no colons)
alice
bob
\end{verbatim}


\subsubsection{Package Declarations}
\label{sec:orgb5d06fc}

The first expression of a file (not counting comments) may be a \textbf{package
declaration}. These have the form \texttt{(package <package-name>)}, where <package-name>
is a symbol.


\subsubsection{Global Names}
\label{sec:org3673a57}

After a namespace has been imported once, its bindings can be referenced even
without importing it explicitly. This is done by using symbols whose names are
structured like with \texttt{/<namespace>:<symbol>}. For example, \texttt{/fn/builtin:map}
refers to the function \texttt{map} in the \texttt{fn/builtin} namespace.


\subsection{{\bfseries\sffamily TODO} Macros}
\label{sec:org725e1c4}

\subsubsection{Macro Basics}
\label{sec:org73c1c41}

\subsubsection{Quasiquotation}
\label{sec:orgb06d932}

\subsubsection{Variable Capture and \texttt{gensym}}
\label{sec:orgf78a814}


\section{Formal Semantics}
\label{sec:org24ca7ef}

This section is a formal description of every type of expression in Fn. It is
currently incomplete and inaccurate. I don't know why you'd want to look at it.

\begin{verbatim}
program ::= expr* expr ::= immediate
        | variable
        | special-form
        | function-call
        | macro-call
\end{verbatim}

\subsection{Immediate Expressions and Variables}
\label{sec:org66b2d79}

Syntax:
\begin{verbatim}
immediate ::= boolean 
          | nil
          | number
          | string
variable ::= non-special-symbol
\end{verbatim}

An immediate expression is a literal representing a constant value. On
evaluation, immediate expressions immediately return the value they represent.

Variables are represented by non-special symbols, (where special symbols are
those naming special forms, boolean values, or nil). If there exists a binding
in the current environment for the provided symbol, then its value is returned.
Otherwise an exception is raised.


\subsection{Special forms}
\label{sec:orgccc45ce}

Special forms are so called because they have different semantics than function
or macro calls.

\subsubsection{and}
\label{sec:orgc6e15b7}
Syntax:
\begin{verbatim}
and-expr ::= "(" "and" expr* ")"
\end{verbatim}

Expressions are evaluated one at a time until a logically false value is
encountered, then returns \texttt{false}. If the end of the list is reached, returns
\texttt{true}.

\subsubsection{cond}
\label{sec:orgcf965a5}
Syntax:
\begin{verbatim}
cond-expr ::= "(" "cond" cond-case+ ")"
cond-case ::= expr expr
\end{verbatim}

For each cond-case, the following is done:
\begin{itemize}
\item evaluate the first expression
\item if the first expression is logically true, return the value of the second
expression
\item otherwise, proceed to the next cond-case.
\end{itemize}

If the end of the list is reached, returns \texttt{nil}.
\subsubsection{def}
\label{sec:orgd2a8583}
Syntax:
\begin{verbatim}
def-expr ::= "(" "def" identifier expr ")"
         | "(" "def" func-proto expr+")"
func-proto ::= "(" identifier param-list ")"
\end{verbatim}

Create a (global) binding in the current namespace. The first syntax binds the
identifier to the value of the expression. The second syntax creates a function
with the specified name and parameter list and the expressions as its body. In
either case, if the identifier is already bound, an exception is raised.

Returns \texttt{null}.

\subsubsection{{\bfseries\sffamily TODO} defmacro}
\label{sec:orgcbb20c2}
Syntax:
\begin{verbatim}
defmacro-expr ::= "(" "defmacro" identifier param-list expr+ ")"
\end{verbatim}

\subsubsection{defn}
\label{sec:org7c526d7}
\subsubsection{do}
\label{sec:org04e81c1}
Syntax:
\begin{verbatim}
do-expr ::= "(" "do" expr* ")"
\end{verbatim}

Evaluates provided expressions one at a time, returning the value of the last
one, or \texttt{null} if no expressions are given.

\subsubsection{do-inline}
\label{sec:orga9c498c}
\subsubsection{{\bfseries\sffamily TODO} dot}
\label{sec:orgd4f17ad}
Syntax:
\begin{verbatim}
dot-expr ::= dotted-symbol
           | "(" "dot" symbol+ ")"
\end{verbatim}

This operator is usually used with the dotted-symbol syntax, e.g. \texttt{table.key}.

The first symbol (leftmost in the inline notation) must name a variable bound to
a table. The next symbol is used as a key to access an element of the table. If
additional symbols are provided, then they are used as keys to recursively
descend into a tree of tables. An exception is raised if one of the keys is
invalid or if an attempt is made to access an object which is not a table.

\subsubsection{dollar-fn}
\label{sec:org6e1a6d5}
Syntax:
\begin{verbatim}
dollar-fn-expr ::= "(" "dollar-fn" expr ")"
               | "$(" expr+ ")"
               | "$[" expr+ "]"
               | "${" expr+ "}"
               | "$`" form
\end{verbatim}

Creates an anonymous function which evaluates the provided expression. With the
"\$" syntax, this is the expression after the dollar sign. (The only expressions
which may follow are parenthesized forms, quasiquote forms, or list/table
expressions).

Within the provided expression, variables named \texttt{\$N} where N is a nonnegative
integer, are bound to the corresponding positional parameters starting from 0.
In addition, \texttt{\$} is bound to the first parameter \texttt{\$0} and \texttt{\$\&} is used for a
variadic parameter.

The parameter list for the created function accepts as many positional
parameters as the highest value of N and a variadic parameter only if \texttt{\$\&}
appears in the expression. (This is accomplished by performing code-walking,
including macroexpansion, before compiling the \texttt{dollar-fn}).

\subsubsection{if}
\label{sec:org4e4e8bd}
Syntax:
\begin{verbatim}
if-expr ::= "(" "if" test-expr expr expr ")"
test-expr ::= expr
\end{verbatim}

Evaluates test-expr. If the result is logically true, evaluates the second
argument, otherwise evaluates final argument, returning the result.

\subsubsection{{\bfseries\sffamily TODO} import}
\label{sec:org4dec73f}
Syntax:
\begin{verbatim}
import-expr ::= "(" "import" import-designator
                             [:as identifier] ")"
import-designator ::= string | symbol
\end{verbatim}

Import bindings from another namespace. Every variable and macro definition from
the target namespace is copied into the current namespace. The newly created
bindings have names of the form \texttt{namespace-name:variable-name}. Since it is
illegal to create variables whose names contain the colon character, this
ensures that no name collisions can occur, provided there is not already a
namespace imported with this name.

If an identifier is provided via the \texttt{:as} argument, then that is used instead of
the namespace name.

When \texttt{import} is used on a namespace that is not already loaded, the interpreter
checks for appropriate files in the search path, then compiles and loads them as
necessary.

\begin{enumerate}
\item Future changes
\label{sec:org670e3b5}

\begin{itemize}
\item Support for unqualified imports will be added (i.e. imports without the \texttt{namespace\textasciitilde{}name:}
prefix).
\item Import may be changed so as to not recursively import imports from the package
named. With current behavior, we could get names like \texttt{ns1:ns2:function} when
\texttt{ns1} imports \texttt{ns2}. This is mostly harmless, but we'd rather not overclutter
namespaces if only for memory concerns. Also, it could definitely cause
trouble if used with unqualified imports.
\item Along with unqualified imports, we may add the notion of exports to
namespaces, so that only certain bindings in a namespace are even made
available to external namespaces. This is probably good for convenience for
users of a library, but it also can give implementors some peace of mind
knowing that their internal functions won't be called directly by user code.
\end{itemize}
\end{enumerate}

\subsubsection{fn}
\label{sec:org0edb310}
Syntax:
\begin{verbatim}
fn-expr ::= "(" "fn" "(" param-list ")" expr+ ")"
\end{verbatim}

Creates a function object using the provided parameter list and function body.

Functions have full variable capture semantics. Among other things, this means
that captured variables are shared among functions, so that if one function
mutates the variable, this change is reflected in the other functions accessing
the variable.

\subsubsection{let}
\label{sec:org86dee5a}
Syntax:
\begin{verbatim}
let-expr ::= "(" "let" let-pair+ ")"
let-pair ::= identifier expr
\end{verbatim}

Extends the current local environment. For each let-pair initially binds the
provided identifier to null. Then, in the order provided, each expression is
evaluated and the binding is updated to the resultant value.

The initial null-binding allows definition of recursive and even mutually
recursive functions. Care must be taken because this null binding will shadow
existing variables with the same name.

Returns null.

\subsubsection{{\bfseries\sffamily TODO} letfn}
\label{sec:orgc57dccf}
\subsubsection{or}
\label{sec:org94bce3a}
Syntax:
\begin{verbatim}
or-expr ::= "(" "or" expr* ")"
\end{verbatim}

Evaluates provided expressions one at a time until a logically true value is
obtained. Then returns \texttt{true}. If the end of the list is reached, returns \texttt{false}.

\subsubsection{quasiquote}
\label{sec:orgf21b880}
Syntax:
\begin{verbatim}
quasiquote-expr ::= "`" form
                | "(" "quasiquote" form ")"
\end{verbatim}

First, creates a fn object corresponding to form just like quote. Before
returning the form, the following transformation is done:
\begin{itemize}
\item The form is walked like a tree.
\item When an unquote-expr is encountered, instead of descending into it, evaluate
its argument and insert the result into the tree at that point.
\item When an unquote-splicing form is encountered, instead of descending into it,
evaluate its argument. If the result is not a list or if this is root of the
tree, raise an error. Otherwise, splice the elements of the list inline into
the tree at this point.
\item Along the way, we keep track of all symbols whose names begin with a hash
character "\#". For each unique hash symbol, a single gensym is created, and
the hash symbols are replaced by the gensyms in the final expansion. For
example, see the following code snippet:
\end{itemize}
\begin{verbatim}
`(#sym1 #sym2 #sym2) ; is the same as
(with (sym1 (gensym)
       sym2 (gensym))
 [sym1 sym2 sym2])
\end{verbatim}
Note that \#-symbols have lexical scope, i.e. they are shared by quasiquote forms
occurring within the outer form.

IMPLNOTE: the semantics of \#-symbols may be too complicated for what they get
us. Might be better to just make the developer deal with \texttt{gensym} directly, or to
make \#-symbols a feature of defmacro.

\subsubsection{quote}
\label{sec:orgcd4124b}
Syntax:
\begin{verbatim}
quote-expr ::= "'" form
           | "(" "quote" form ")"
\end{verbatim}

Returns the syntactic form as an fn object (a tree of atoms and lists).

\subsubsection{unquote}
\label{sec:orgfe08fb0}
Syntax:
\begin{verbatim}
unquote-expr ::= "," expr
             | "(" "unquote" expr ")"
\end{verbatim}
Emits an error unless encountered within a quasiquote form.

\subsubsection{unquote-splicing}
\label{sec:org3e26911}
Syntax:
\begin{verbatim}
unquote-splicing-expr ::= ",@" expr | "(" "unquote-splicing" expr ")"
\end{verbatim}
Emits an error unless encountered within a quasiquote form.

\subsubsection{{\bfseries\sffamily TODO} set!}
\label{sec:orgdce4a8b}
Syntax:
\begin{verbatim}
set!-form ::= "(" "set!" place expr ")"
place ::= identifier 
      | dot-expr
      | get-form
get-form ::= "(" "get" expr+ ")"
\end{verbatim}

\subsubsection{with}
\label{sec:org3b2c38d}
Syntax:
\begin{verbatim}
with-expr ::= "(" with-bindings expr+ ")"
with-bindings ::= "(" (id expr)* ")"
\end{verbatim}

Behaves like \texttt{let}, but rather than operating on the enclosing lexical
environment, instead creates a new child environment and adds bindings to that,
then evaluates the provided expressions in the newly created environment.

Note that this is how \texttt{let} works in most LISP-like languages.


\subsection{Function Calls}
\label{sec:orgb336eed}

Syntax:
\begin{verbatim}
function-call ::= "(" func argument-list ")"
\end{verbatim}
where \texttt{func} may be any expression other than a reserved symbol.

First, the function and then all the arguments are evaluated from left to right.

Arguments are bound to parameters as follows:

Positional arguments are bound to the function's parameters in the order
provided. If there are more positional arguments than parameters, a list of the
extras are bound to the variadic list parameter if one exists. If not, an error
is generated.

After this, keyword arguments are bound to parameters by name. If two keyword
arguments have the same name, an error is raised. If the name isn't one of the
function's parameters, or if it names a parameter already provided by a
positional argument, it is added to the variadic table parameter if one exists.
If not, an error is raised.

If any parameters without default values remain unbound, an error is raised.

Then, the function is called. Foreign functions have behavior determined by the
external code they call. Ordinary functions work by switching back to the
lexical environment in which they were created, binding parameters as local
variables, and executing the function body.


\subsection{Macro Calls}
\label{sec:orga472829}

Syntax:
\begin{verbatim}
macro-call ::= "(" macro-name form* ")"
macro-name ::= identifier | dot-expr
\end{verbatim}

The expressions for macro arguments aren't evaluated, but are converted to data
and passed to the macro function as arguments. The resultant value is treated as
code and evaluated.

In order to prevent ambiguity, macros do not recognize keyword arguments. A
keyword will be passed to the macro as a positional argument containing the
keyword symbol.


\section{Lexical Analysis}
\label{sec:orgaaae3dc}

The component which processes Fn source into lexical tokens is called the
\textbf{scanner}. Conceptually, we put in a sequence of bytes and get back a sequence
of tokens.

Tokens are dividide into two groups: fixed width and variable width. The fixed
width tokens are:
\begin{description}
\item[{'('}] left paren
\item[{')'}] right paren
\item[{'['}] left bracket
\item[{']'}] right bracket
\item[{'\{'}] left brace
\item[{'\}'}] right brace
\item[{'$\backslash$''}] quote
\item[{'`'}] backtick
\item[{',@'}] comma at
\item[{','}] comma
\item[{'\$`'}] dollar backtick
\item[{'\$('}] dollar paren
\item[{'\$['}] dollar bracket
\item[{'\$\{'}] dollar brace
\end{description}

Fixed width tokens are generated when the scanner encounters one of the
corresponding quoted strings above.

Variable width tokens are:
\begin{description}
\item[{<number>}] numeric literal
\item[{<string>}] string literal
\item[{<symbol>}] symbol
\item[{<dot>}] dot expression (2 or more symbols separated by '.')
\end{description}

\subsection{Comments}
\label{sec:org08d4324}

A comments begins with the unescaped character \texttt{';'} and end at the end of the
line.

Comments are skipped over without generating a token.

\subsection{Numbers, Symbols, and Dots}
\label{sec:org4232e10}

Numbers are defined according to regular expressions as in the following
pseudo-BNF grammar:

\begin{verbatim}
<number> ::= "[+-]?(<dec>|<hex>)"
<dec>    ::= "[0-9]+\.?<exp>?"
           | "[0-9]*\.[0-9]+<exp>?"
<exp>    ::= "[eE][+-]?[0-9]+"
<hex>    ::= "0[Xx][0-9A-Fa-f]+\.?"
           | "0[Xx][0-9A-Fa-f]*\.[0-9A-Fa-f]"
\end{verbatim}

A symbol is defined to be a sequence of symbol characters and escaped characters
which is not a number. Symbol characters are all printable ASCII characters
other than whitespace or those contained in the string \texttt{";()\{\}[]\textbackslash{}"\textbackslash{}\textbackslash{}'`,."}. An
escaped character is a sequence of two characters, the escape \texttt{'\textbackslash{}\textbackslash{}'} followed by
an arbitrary character. When the symbol is internalized, the escape character is
ignored, so its name will contain only the second character of the escape
sequence.

If an escape character is followed by EOF, a scanning error is raised.

Note that by injecting an escape character, one may cause numbers to be treated
as symbols. Adding an escape in front of any normal character normally no
effect, but in this case, it causes the number reader to fail, so the number's
characters will read as a symbol.

Finally, a dot token is a symbol token which contains the dot character \texttt{'.'},
subject to some additional restrictions. Dots can not occur in the first or last
position of the string, and it can not contain successive dots. In addition, the
substring before the first dot may not be a syntactically valid number. If any
of these conditions is violated, a scanning error is raised.

\subsection{{\bfseries\sffamily TODO} String Literals}
\label{sec:org01888e7}

String scanning starts when the scanner encounters the (unescaped) character
\texttt{'"'} and ends when it encounters an unescaped \texttt{'"'}. Along the way, all bytes
encountered are read verbatim, except for the escape character '$\backslash$\', which is
followed by an escape sequence. The entire escape sequence is read into the
string according to the following table \footnote{These string escapes are mainly the same as the ones in C.}:
\begin{description}
\item[{'$\backslash$''}] single quote
\item[{'$\backslash$"'}] double quote
\item[{'$\backslash$?'}] question mark
\item[{'\a'}] ASCII bell
\item[{'\b'}] backspace
\item[{'\f'}] form feed
\item[{'\n'}] newline
\item[{'\r'}] ASCII carriage return
\item[{'\t'}] tab
\item[{'\v'}] vertical tab
\item[{'\NNN' (NNN is a 1- to 3- digit octal number)}] byte NNN (octal)
\item[{'\xNN' (NN any two-digit hex number)}] the byte NN (hexadecimal)
\item[{'\uC' (C any 4-digit hex unicode code point)}] unicode code point (2 bytes)
\item[{'\UC' (C any 8-digit hex unicode code point)}] unicode code point (4 bytes)
\end{description}

\subsection{Source Encoding}
\label{sec:org523c108}

Currently, Fn only supports ASCII encoded text files. Behavior on other/extended
encodings is undefined. In the future, Fn will be extended so that UTF-8
characters can appear in strings and symbols.


\section{{\bfseries\sffamily TODO} Built-in Functions}
\label{sec:org9c50f32}

Fn provides a number of built-in functions in the namespace \texttt{fn.builtin}.
Whenever a new namespace is created, it automatically inherits all the bindings
from \texttt{fn.builtin}, so these bindings are always available as global variables.

Builtin functions are split into primitive and nonprimitive functions. This
classification is not overly rigorous, but the rule is that primitive functions
expose core language functionality which cannot be used any other way. On the
other hand, all nonprimitive functions could theoretically be implemented in Fn
source code by using the primitives.

\subsection{Primitive Functions}
\label{sec:org4e804c2}

\begin{description}
\item[{apply (obj arg0 arg1 \& args)}] Call obj as a function. Positional arguments
for the call are generated by taking a list of all but the last two arguments
(to apply), and concatenating that with the second-to-last argument, which
must be a list. The last argument (to apply) is a table of keyword arguments.

\item[{gensym ()}] Generate a symbol with a guaranteed unique id. This is used for
macro writing.
\item[{symbol-name (x)}] get the name of a symbol as a string

\item[{length (obj)}] Depending on the type of obj, returns
\begin{itemize}
\item the length of a string in bytes,
\item the number of elements in a list, or
\item the number of keys in a table or namespace.
\end{itemize}
\item[{concat (seq0 \& seqs)}] Concatenate strings or lists. All the arguments must
be of the same type.
\item[{nth (n seq)}] Get the nth element of a list or string. If seq is a string the
result will be a string of length 1 containing the (n-1)th byte of the string.

\item[{=}] check for equality
\item[{same?}] check for equality
\item[{not}] logical not
\end{description}


Type Checkers:

\begin{description}
\item[{bool?}] 

\item[{function?}] 

\item[{int?}] 

\item[{list?}] 

\item[{namespace?}] 

\item[{number?}] 

\item[{null?}] 

\item[{string?}] 

\item[{symbol?}] 

\item[{table?}] 
\end{description}

Lists and Strings:
\begin{description}
\item[{List (\& objs)}] Create a list of the given objects
\item[{empty? (obj)}] Equivalent to \texttt{(= obj [])}.
\item[{cons (hd tl)}] Create a new list by prepending hd to tl.
\item[{head (list)}] get the first element of a list. Error on empty.
\item[{tail (list)}] drop the first element of a list. Returns empty on empty.
\end{description}

Tables and Namespaces:
\begin{description}
\item[{Table (\& kv-pairs)}] Create a table. The argument list is a sequence
of pairs consisting of keys followed by values.
\item[{get (key obj)}] access a field from a table or namespace
\item[{get-keys (obj)}] get a list of keys from a table or namespace
\item[{has-key? (obj key)}] get a list of keys from a table or namespace
\end{description}

Arithmetic:
\begin{description}
\item[{+}] 

\item[{-}] 

\item[{*}] 

\item[{/}] 

\item[{**}] 

\item[{<}] 

\item[{>}] 

\item[{<=}] 

\item[{>=}] 

\item[{floor}] 

\item[{ceil}] 
\end{description}

\subsection{Nonprimitive functions}
\label{sec:orgf3599f1}

(Note: a user type can implement any or all of these functions by adding methods
for them. Sorry, that isn't documented yet).

All of these are non-destructive.

\begin{itemize}
\item has-key? (table)

\item concat (\& seqs)
\item reverse (seq)

\item insert (elt n seq)
\item append (elt seq)
\item prepend (elt seq)
\item sort (seq (ascending true))
\item sort-by (fun seq (ascending true))

\item head (seq)
\item tail (seq)
\item nth (n seq)

\item take (test seq)
\item drop (test seq)
\item take-while (test seq)
\item drop-while (test seq)
\item split-at (n seq)
\item split-when (test seq)

\item group (n seq)
\item group-by (key seq)
\item subseq (start end seq)

\item dedup (seq) [remove duplicates]
\item replace (n elt seq)

\item empty? (seq)
\item contains? (seq)

\item length< (seq n) [compare lengths w/o nec. computing the thing]
\item length> (seq n)
\item length<= (seq n)
\item length>= (seq n)

\item map (fun seq0 \& seqs)
\item fold (fun init seq0 \& seqs)
\item filter (test seq)
\item every? (test seq)
\item any? (test seq)
\end{itemize}




\section{Future Extensions}
\label{sec:org72dcd46}

The version of Fn described in this document is version 0.1. The language \textbf{will}
change before version 1.0. In particular, I have to, \textbf{have to}, add pattern
matching.

Items marked with (Proposed) below are things that might not make it into the
final language.

\subsection{Pattern Destructuring in \texttt{def} and \texttt{let}}
\label{sec:orgb8d083e}

Once pattern matching is added to Fn, \texttt{let} and \texttt{def} will be able to do
automatic destructuring on the pattern to bind variables. For instance,
\begin{verbatim}
(def [x y] [1 2])
\end{verbatim}
binds \texttt{x} to \texttt{1} and \texttt{y} to \texttt{2}.

When matching fails, an error will be raised.


\subsection{(Proposed) \texttt{\#}-Syntax for Collections}
\label{sec:org6d1883e}

So, we turn \texttt{\#} into a syntax character and give \texttt{\#()}, \texttt{\#[]}, \texttt{\#\{\}}, etc. syntactic
meanings.

There are a couple of options for how to do this:
\begin{itemize}
\item \texttt{\#()} could expand to \texttt{(values)} (if we implement multiple values as proposed), or
to a set or tuple type.
\item \texttt{\$\{\}} could expand to \texttt{(Set)} or \texttt{(Vector)}, or if we add typed tables, \texttt{\#\{type ...\}}
could be used to construct types (e.g. \texttt{(construct type ...)}).
\item \texttt{\#[]} could expand to \texttt{(Vector)} or \texttt{(values)}.
\item Alternatively, if we lean into lazy data structures, then the \texttt{\#} structures
could be the respective lazy equivalents
\end{itemize}

I'm most likely to set \texttt{\#[]} to \texttt{(Vector ...)}. Using \texttt{\#\{\}} for \texttt{(Set ...)} would also
make sense, but if typed tables get added, those will be more important for
sets. If multiple values get added, then the \texttt{\#()} syntax should really be used
for that.

Drawbacks: this would make hash into a special character as opposed to just a
symbol constituent.


\subsection{Global Names for Definitions}
\label{sec:org9fae995}

A global variable name consists of a forward slash, the full namespace name, a
colon, and an identifier in that namespace, in that order. For example:
\begin{verbatim}
/fn/builtin:+
\end{verbatim}
is the global name for the builtin \texttt{+} function.

This is implemented as follows:
\begin{itemize}
\item make it illegal to name namespaces beginning with a slash
\item make it illegal to create variables whose names contain colons.
(Alternatively, can make it illegal to create variables whose names start with
a slash).
\item variable names that have this syntax are evaluated a little differently
\end{itemize}

To get the global name of a symbol, we'll provide a builtin function called
\texttt{global-name}, e.g. \texttt{(global-name '+)} is \texttt{'/fn/builtin:+}.


\subsection{(Proposed) \texttt{,:}-Syntax for Global Names in Quasiquote}
\label{sec:orgc8beef5}

The syntax \texttt{,:var} expands \texttt{(unquote (global-name 'var))}. This will be necessary
for all non-builtin functions included in macro expansions, (and should be used
for builtin functions too).


\subsection{(Proposed) Multiple Return Values}
\label{sec:org223188a}

Idk, these are useful in Matlab and Haskell has this sort of thing going on with
its tuples.

So we create a multiple values object with \texttt{(values x y ...)}. These objects are
tricky; as soon as they get used they collapse to a single value (in this case
the result of expression \texttt{x}).

However, when a multiple values object is returned to a \texttt{let} or \texttt{def}, then we can
use destructuring to access the additional values. For example:
\begin{verbatim}
(defn my-fun () (values 1 2 3))
(def x (my-fun)) ; sets x = 1
(def (values y z) ; sets y = 1, z = 2
(def (values y z w) ; sets y = 1, z = 2, w = 3
\end{verbatim}

Together with \#-syntax for values, this might be a very useful feature. It would
look like this:
\begin{verbatim}
(def #(y z) (my-fun)) ; sets y = 1, z = 2
(def (values y z w) (my-fun)) ; sets y = 1, z = 2, w = 3
\end{verbatim}


\subsection{Schemas, Metatables, and Table Calls}
\label{sec:org34618cf}

Tables are gonna get a metatable like in Lua. This will simply be a second,
invisible table, accessible via a \texttt{metatable} builtin function.

The \texttt{metatable} may contain these two things (more may be added):
\begin{itemize}
\item an \textbf{on-call function}. If set, this allows the table to be called like a
function by invoking the on-call function on the given arguments.
\item a \textbf{schema} for the table
\end{itemize}

The on-call function is self-explanatory:
\begin{verbatim}
(def tab { 'value 27 })
(set! (get (metatable tab) 'on-call) 
      (fn (x) (+ x tab.value)))
(tab 42) ;=> 69
\end{verbatim}

Schemas are objects describing the following information:
\begin{itemize}
\item a schema name that can be used to do pattern matching on this kind of object
\item optional type annotations for values in the table
\item pattern matching functions
\item (Proposed) protocols implemented by the table. (Note: this may belong directly
in the metatable?)
\end{itemize}

So, giving a table a schema provides:
\begin{itemize}
\item data validation
\item accessors
\item pattern matching
\end{itemize}
All of this with very little overhead.

So giving a table a schema is a big deal. It basically promotes the the object
from a table to a more complex type. For this reason, we call tables with
non-nil schemas \textbf{structures}. It is intended that structures will be used
extensively in Fn as the main way to organize data.

Through use of variable capture, functions contained in a table may access or
mutate that table. This makes it possible to implement full, stateful
object-oriented programming using structures containing functions, although I
don't know why you'd want to.

However, we functional programmers still benefit from structures containing
functions. In particular, this is a handy way to simultaneously save typing and
avoid namespace pollution. For instance, consider:
\begin{verbatim}
;; database structure has a send function
(my-db.query my-query)
;; vs. a global function to do it
(db-query my-db my-query)
\end{verbatim}
Moreover, we see this gives us something Python calls "duck typing". Any
table with an appropriate query function will work in the first expression
above. I think that's pretty neat.

You could even implement ADTs a la Haskell using schemas, but I don't expect
they'd be nearly as useful without that sweet, sweet type system.


\subsection{(Proposed) Table Immutability Rules}
\label{sec:orge599d70}

It would be really great to have some guarantee of a table's immutability.

Mutable tables are great for initialization and certain types of imperative
algorithms. However, immutable tables are better for functional programming, and
we like that.

I propose the following changes:
\begin{itemize}
\item tables are immutable by default
\item mutable tables may be created via \texttt{(M-Table ...)}, and have mutable metatables
\item builtin functions are provided to copy mutable tables to immutable ones and
vice versa
\end{itemize}

In addition, it may be possible to allow mutable tables to be made immutable in
certain situations, in order to save the copying operation. In order to prevent
this from wreaking havoc, it might be best to provide some builtin function that
exposes this functionality. E.g. \texttt{(build-table x)} where x is a function that
takes a mutable table, mutates it, and returns it at the end. Provided the
return value is the same table as was put in, it is then converted to a mutable
table directly, with no copying.
\end{document}